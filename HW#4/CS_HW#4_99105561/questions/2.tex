\subsection*{الف}
برای تابع توزیع احتمال خرید میوه‌ها با توجه به جدول داده شده خواهیم داشت:
\begin{flalign*}
    P(x = i) = 
    \begin{cases}
        \frac{5}{11} \quad i = 1 \\
        \frac{3}{11} \quad i = 2 \\
        \frac{2}{11} \quad i = 3 \\
        \frac{1}{11} \quad i = 4
    \end{cases} &&
\end{flalign*}

\subsection*{ب}
حال با توجه به روش
\LR{Inverse-transform}
ابتدا به دست می‌آوریم که هر عدد بین کدام بازه‌ها مطلق به کدام
$x_i$
است.
\begin{flalign*}
    x_i = 
    \begin{cases}
        1 \quad 0 \leq R_i \leq \frac{5}{11} \\
        2 \quad \frac{5}{11} \leq R_i \leq \frac{8}{11} \\
        3 \quad \frac{8}{11} \leq R_i \leq \frac{10}{11} \\
        4 \quad \frac{10}{11} \leq R_i \leq 1
    \end{cases} &&
\end{flalign*}
حال برای واریته متناظر با اعداد داده شده خواهیم داشت.
\begin{flalign*}
    R_1 &= 0.3 \implies 0 \leq R_1 \leq \frac{5}{11} \implies x_1 = 1 \\
    R_2 &= 0.45 \implies 0 \leq R_2 \leq \frac{5}{11} \implies x_2 = 1 \\
    R_3 &= 0.6 \implies \frac{5}{11} \leq R_3 \leq \frac{8}{11} \implies x_3 = 2 \\
    R_4 &= 0.8 \implies \frac{8}{11} \leq R_4 \leq \frac{10}{11} \implies x_4 = 3 \\
    R_5 &= 0.95 \implies \frac{10}{11} \leq R_5 \leq 1 \implies x_5 = 4 &&
\end{flalign*}