دو مثالی که می‌توان از توزیع
Weibull
و کاربردهای آن در شبیه‌سازی زد به این صورت است:
\begin{itemize}
    \item
    یکی از کاربردهای این توزیع استفاده آن در استحکام سنجی موزایک‌های تولیدی است. برای اینکار پارامترهایی مختلف همانند سختی، دما را در نظر گرفته و یک حد پایینی برای مقاومت و استحکام موزایک‌ها با استفاده از شبیه‌سازی می‌توان به دست آورد.
    \item
    کاربرد دیگر این توزیع، شبیه‌سازی حجم آب ورودی سدها و سرعت آنهاست. با توجه به عوامل مختلفی همانند ورودی‌های سد، فصل از سال می‌توان از این توزیع استفاده نمود و کمکی نیز برای پیش‌بینی حجم برق تولیدی آن سد خواهد بود.
\end{itemize}

از کاربردهای توزیع
\LR{Lognormal}
می‌توان به این دو مورد نیز اشاره کرد. 
\begin{itemize}
    \item
    در سیستم مدیریت بانکی برای شبیه‌سازی ورودی حساب‌ها، در این حالت از آنجایی که بیشتر سرمایه در دسترس اندکی از آدم‌ها قرار دارد این توزیع مورد استفاده قرار می‌گیرد.
    \item
    تبادلات فایل در یک شبکه نیز به همین صورت بوده، اکثر فایل‌های انتقالی در اندازه‌های کوچک بوده اما تعداد کمی فایل نیز وجود دارد که اندازه بسیار زیادی داشته باشند.
\end{itemize}