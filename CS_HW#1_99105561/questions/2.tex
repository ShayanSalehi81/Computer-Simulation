\subsection*{الف}
رستوران:

موجودیت‌های ممکن شامل مشتری و آشپز بوده، برای ویژگی‌ها می‌توان سفارشات و سرعت پخت و پز را نام برد. برای فعالیت‌های نیز آماده‌سازی و انتخاب غذا را خواهیم داشت. برای رویدادهای ممکن می‌توان پایان غذای مشتری، ورود مشتری جدید و آماده‌سازی غذا را گفت. در نهایت برای متغییرهای حالت تعداد آشپزهای مشغول و وضعیت مشتری‌ها را خواهیم داشت.

\subsection*{ب}
شرکت تاکسی تلفنی:

همانند قسمت بالا برای موجودیت‌ها مسافر و راننده؛ برای ویژگی‌ها سرعت حرکت، مبدا، ترافیک و مقصد؛ برای فعالیت‌ها سفارش مسافر و سفر کردن: برای رویدادها رسیدن به مبدا، مقصد و درخواست تاکسی و در نهایت برای متغیرهای حالت، تعداد مشتری در انتظار و تعداد راننده در حال حرکت را خواهیم داشت.

\subsection*{ج}
سوپر مارکت:


در اینجا برای موجودیت‌ها کالا و خریدار، فروشنده؛ برای ویژگی‌ها مقدار پول برای هزینه کردن، تعداد کالا؛ برای فعالیت‌ها خرید کالا و حساب تمامی خریدها: برای رویدادها اتمام خرید مشتری، ورود خریدار و محاسبه هزینه خرید توسط فروشنده؛ در نهایت برای متغیرهای حالت، تعداد مشتری در انتظار، تعداد فروشنده را خواهیم داشت:

\subsection*{د}
رسانه اجتماعی:

موجودیت‌ها شامل کاربر، کانال، پیام و گروه. ویژگی‌ها شامل اعضای گروه و کانال، تعداد و طول پیام‌ها. برای فعالیت‌ها ساخت گروه یا کانال و ارسال پیام، برای رویدادها جوین دادن به گروه و کانال و جابه‌جای پیام، در نهایت برای متغیرهای حالت تعداد اعضای گروه یا کانال و تعداد پیام را خواهیم داشت.
